%% -!TEX root = AUTthesis.tex
% دانشکده، آموزشکده و یا پژوهشکده  خود را وارد کنید
\faculty{دانشکده ریاضی و علوم کامپیوتر}
% گرایش و گروه آموزشی خود را وارد کنید
\department{علوم کامپیوتر}
% عنوان پایان‌نامه را وارد کنید
\fatitle{پیاده سازی بازی \lr{Pacman}
\\[.75 cm]}
% نام استاد(ان) راهنما را وارد کنید
\firstsupervisor{جناب آقای دکتر قطعی}
%\secondsupervisor{استاد راهنمای دوم}
% نام استاد(دان) مشاور را وارد کنید. چنانچه استاد مشاور ندارید، دستور پایین را غیرفعال کنید.
%\firstadvisor{نام کامل استاد مشاور}
%\secondadvisor{استاد مشاور دوم}
% نام نویسنده را وارد کنید
\name{ مهدی}
% نام خانوادگی نویسنده را وارد کنید
\surname{ عباسعلی پور}
%%%%%%%%%%%%%%%%%%%%%%%%%%%%%%%%%%
\thesisdate{آبان ماه 1402}
% image path
\graphicspath{ {./img} }
% چکیده پایان‌نامه را وارد کنید
\fa-abstract{
در این پروژه قصد داریم تا بازی 
\lr{Pacman}
را با استفاده از زبان برنامه نویسی پایتون پیاده سازی نماییم . به این صورت که در زمین بازی دو روح در هر گام به صورت تصادفی حرکت می نمایند و عامل هوشمند باید با الگوریتم 
\lr{min-max}
به صورت صحیح 
\lr{pacmn}
را هدایت نماید تا به روح ها برخورد نکند و در ضمن با کمترین حرکت بیشترین نقاط را بخورد .
}


% کلمات کلیدی پایان‌نامه را وارد کنید
\keywords{بازی \lr{pacman} , الگوریتم \lr{min-max}}
\renewcommand{\bibname}{مراجع}



\AUTtitle
%%%%%%%%%%%%%%%%%%%%%%%%%%%%%%%%%%
\vspace*{7cm}
\thispagestyle{empty}
\begin{center}
\includegraphics[height=5cm,width=12cm]{besm}
\end{center}